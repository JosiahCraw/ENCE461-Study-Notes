Digital Signal Processing (DSP) does performs signal processing in the digital domain. This approach
is simpler than analog signal processing. This sort of processing is well suited for speech and
language processing, image processing etc.

\subsection{System Design Considerations}
\subsubsection{Hardware}
When selecting hardware ensure that the DSP has the appropriate fixed/floating point data unit,
adequate data width and processing speed for the desired application. Also the sampling rate may be
either fixed or variable. The available interfaces are important, such as serial and parallel
interface compatibility. Finally the memory types are an important consideration for hardware
selection.

\subsubsection{Software}
When making software design decisions ensure the appropriate language/group of languages are used to
best fit the desired use case. Also consider any operating systems and any debugging or testing
toolchains that may be required.\\

\noindent DSP's are best suited for low complexity/high sample rate data collection and processing.

\begin{table}[H]
\caption{MCU vs. DSP Comparison Table}
\label{table:mcu-v-dsp}
\begin{center}
    \begin{tabular} { | m{4cm} | m{4cm} | m{4cm} | }
        \hline
        \empty & Microcontrollers (MCUs) & Digital Signal Processors\\
        \hline
        Functionality & General purpose, more suitable for sensing and control applications & More
        specialised, architecture and instruction set more suitable for signal processing\\
        \hline
        Power Consumption & Suitable for very low power applications & Typically run at full power\\
        \hline
        On-chip Peripherals & Typically have extensive inbuilt support & Most peripherals are not
        supoorted\\
        \hline
        Architecture & RISC-based with high performance extensions & Optimised instruction and
        architecture for executing repetitive algorithms\\
        \hline
        Ease of Use & Short development with good documentation and tools & Longer development time
        as there is more of an optimisation focus\\
        \hline

    \end{tabular}
\end{center}
\end{table}


\subsection{Common DSP Features}
DSP's usually have features such as, multiple bus architectures eg. Harvard, modified Harvard etc.,
fast multiply and accumulate (MAC) and dot product processing. Typically DSP's also execute one
instruction per cycle and have very fast clock rates. Also present are special address modes eg.
circular addressing, specialised execution and control (no loop overhead, and a delayed branch
instruction) and specialised peripherals \& I/O devices eg. communications between DSP and CODECs,
multiple channel serial interfaces and DMA modules. Solutions such as DSP+ARM exist, with is a
combination of ARM core/s for command and control and a DSP for real-time processing.

\subsection{MAC}
Multiply and accumulate (MAC) is a core DSP operation and hence must be very efficient examples of
multiply and accumulate functions are:

\begin{itemize}
    \item Convolution
    \item Correlation (Auto and Cross)
    \item Fourier and Inverse Fourier transforms
    \item $c_{i}=\sum_{k=0}^{N}a_{i} \dot b_{i}$
    \item Digital/adaptive filtering
    \item Neural Networks
\end{itemize}

DSP's typically have dedicated MAC modules to enable extreemly efficent MAC an example of this is
shown in Figure \ref{fig:mac-unit}.

\begin{figure}[H]
\begin{center}
    \begin{tikzpicture}
        \draw (0,-1) node[boxes] (m0) {$x(n-k)$};
        \draw (4,-1) node[boxes] (m1) {$b_k$};
        \draw (-1,-2) -- (1,-2) -- (2,-3) -- (3,-2) -- (5,-2) -- (3,-4) -- (1,-4) --
        (-1,-2);

        \draw (2,-3.5) node[] {Multiplier};

        \draw (2,-6) node[xwboxes] (acc) {Accumulator};

        % Arrows
        \draw[arrow] (m0.south) -- (0,-2);
        \draw[arrow] (m1.south) -- (4,-2);

        \draw[arrow] (2,-4) -- (acc.north);
        \draw[arrow] (acc.south) -- (2,-8) node[below] {Output};
        \draw (2,-7) -| (-3,-5);
        \draw[arrow] (-3,-5) -| (-1,-5.5);
    \end{tikzpicture}
\end{center}
\caption{DSP Multiply and Accumulate (MAC) Module}
\label{fig:mac-unit}
\end{figure}


\subsubsection{FIR Filtering}
Finite Impulse Response filtering (FIR) is a digital filtering method, an execution of which can be
found in Figure \ref{fig:fir-dsp}

\begin{figure}[H]
\begin{center}
\begin{adjustbox} {width=\textwidth}
    \begin{tikzpicture}
        \draw (2,0) node[boxes] (z0) {$z^{-1}$} (6,0) node[boxes] (z1) {$z^{-1}$} (10,0) node[boxes]
        (z2) {$z^{-1}$};

        \draw (0,-1.5) circle (0.5cm);
        \draw[arrow] (-1.5,-1.5) node[left] {$h(0)$} -- (-0.5,-1.5);
        \draw (0,-1.5) node[] {$\times$};

        \draw (4,-1.5) circle (0.5cm);
        \draw[arrow] (2.5,-1.5) node[left] {$h(1)$} -- (3.5,-1.5);
        \draw (4,-1.5) node[] {$\times$};
        \draw (4,-3.5) circle (0.5cm);
        \draw (4,-3.5) node[] {$+$};

        \draw (8,-1.5) circle (0.5cm);
        \draw[arrow] (6.5,-1.5) node[left] {$h(2)$} -- (7.5,-1.5);
        \draw (8,-1.5) node[] {$\times$};
        \draw (8,-3.5) circle (0.5cm);
        \draw (8,-3.5) node[] {$+$};

        \draw (12,-1.5) circle (0.5cm);
        \draw[arrow] (10.5,-1.5) node[left] {$h(3)$} -- (11.5,-1.5);
        \draw (12,-1.5) node[] {$\times$};
        \draw (12,-3.5) circle (0.5cm);
        \draw (12,-3.5) node[] {$+$};

        \draw[arrow] (-1.5,0) node[left] {$x(n)$} -- (z0.west); 
        \draw[arrow] (-1.5,0) -| (0,-1);
        \draw[arrow] (0,-2) |- (3.5,-3.5);

        \draw[arrow] (z0.east) -- (z1.west);
        \draw[arrow] (z0.east) -| (4,-1);
        \draw[arrow] (4,-2) -- (4,-3);

        \draw[arrow] (z1.east) -- (z2.west);
        \draw[arrow] (z1.east) -| (8,-1);
        \draw[arrow] (8,-2) -- (8,-3);

        \draw[arrow] (z2.east) -| (12,-1);
        \draw[arrow] (12,-2) -- (12,-3);
        
        \draw[arrow] (4.5,-3.5) -- (7.5,-3.5);
        \draw[arrow] (8.5,-3.5) -- (11.5,-3.5);
        \draw[arrow] (12.5,-3.5) -- (14,-3.5) node[right] {$y(n)$};
        
        \draw (0,-4) node[below] {Multiply};
        \draw (4,-4) node[below] {Multiply \& Add};
        \draw (8,-4) node[below] {Multiply \& Add};
        \draw (12,-4) node[below] {Multiply \& Add};
    \end{tikzpicture}
\end{adjustbox}
\end{center}
\caption{FIR Filter DSP Operations}
\label{fig:fir-dsp}
\end{figure}


This results in the following operations:

\begin{center}
\begin{tabular} {l l l}
    Fir: & IN & PORTX\\
    & MPY & *x-, *h+,AC0\\
    & MAC & *x-, *h+,AC0\\
    & MAC & *x-, *h+,AC0\\
    & MAC & *x-, *h+,AC0\\
    & MAC & *x-, *h+,AC0\\
    & OUT & PORTY
\end{tabular}
\end{center}

\subsection{Fixed Point Data}
In a fixed point DSP data is represented by a binary fraction $\in (-1, 1)$ this method reduces the
dynamic range for multiplication eg. $0.9\times 0.9 = 0.81$ to maintain the original bit width the
output value is concatenated to $0.8$ to encode the numbers two compliment is used as it allows sign
coding. When multiplying negative values `1's are appended to the front. The result is output in
two's compliment.

\noindent The binary mapping to decimal values are:
\begin{equation}
\begin{bmatrix}
    -1 & \frac{1}{2} & \frac{1}{4} & \frac{1}{8}\\
\end{bmatrix}
\end{equation}

\begin{equation}
\begin{tikzpicture}[baseline={([yshift=-.5ex]current bounding box.center)}]
    \node[matrix of math nodes] (fixed) {
        \node[fill=red!12](fixed-1-1){1};&\node[fill=red!12]{1};&\node[fill=red!12]{1};&\node[fill=red!12]{1};&
        \node[fill=red!12]{1};&\node[fill=red!12]{1};&\node[fill=red!12]{0};&\node[fill=red!12](fixed-1-8){1};\\
        \empty & \empty & \times & \empty & 0&1&0&\node[fill=red!12](fixed-2-8){0};\\
        \node[fill=blue!12](fixed-3-1){0};&\node[fill=blue!12]{0};&\node[fill=blue!12]{0};&\node[fill=blue!12]{0};&
           \node[fill=blue!12]{0};&\node[fill=blue!12]{0};&\node[fill=blue!12]{0};&\node[fill=blue!12](fixed-3-8){0};\\
        0&0&0&0&0&0&0\\
        1&1&1&1&0&1\\
        0&0&0&0&0\\
        \textbf{\sout{1}}&\textbf{1.}&\textbf{1}&\textbf{1}&\textbf{0}&\textbf{\sout{1}}&\textbf{\sout{0}}&\textbf{\sout{0}}\\
    };
    \draw (fixed-1-8.east) node[right] {$\gets \frac{-3}{8}$};
    \draw (fixed-2-8.east) node[right] {$\gets \frac{1}{2}$};
    \draw (fixed-7-1.west) node[left] {Left shift off first value $\to$};
    \draw (fixed-3-1.north west) -- (fixed-3-8.north east); 
    \draw (fixed-7-1.north west) -- (fixed-7-8.north east);
    \draw[arrow] (fixed-7-2.south) -- ++(0,-1) node[below] {6 Digits after D.P};
\end{tikzpicture}
\label{eqn:fixed-point}
\end{equation}

The following fixed point calculation is $\frac{3}{8} \times \frac{1}{4}$

\begin{equation}
\begin{tikzpicture}[baseline={([yshift=-.5ex]current bounding box.center)}]
    \node[matrix of math nodes] (fixed) {
        \node[fill=red!12](fixed-1-1){0};&\node[fill=red!12]{0};&\node[fill=red!12]{0};&\node[fill=red!12]{0};&
            \node[fill=red!12]{0};&\node[fill=red!12]{0};&\node[fill=red!12]{1};&\node[fill=red!12](fixed-1-8){1};\\
        \empty & \empty & \times & \empty & 0&0&1&\node[fill=red!12](fixed-2-8){0};\\
        \node[fill=blue!12](fixed-3-1){0};&\node[fill=blue!12]{0};&\node[fill=blue!12]{0};&\node[fill=blue!12]{0};&
           \node[fill=blue!12]{0};&\node[fill=blue!12]{0};&\node[fill=blue!12]{0};&\node[fill=blue!12](fixed-3-8){0};\\
        0&0&0&0&0&1&1\\
        0&0&0&0&0&0\\
        0&0&0&0&0\\
        \textbf{\sout{0}}&\textbf{0.}&\textbf{0}&\textbf{0}&\textbf{0}&\textbf{\sout{1}}&\textbf{\sout{1}}&\textbf{\sout{0}}\\
    };
    \draw (fixed-1-8.east) node[right] {$\gets \frac{3}{8}$};
    \draw (fixed-2-8.east) node[right] {$\gets \frac{1}{4}$};
    \draw (fixed-7-1.west) node[left] {Left shift off first value $\to$};
    \draw (fixed-3-1.north west) -- (fixed-3-8.north east); 
    \draw (fixed-7-1.north west) -- (fixed-7-8.north east); 
    \draw[arrow] (fixed-7-2.south) -- ++(0,-1) node[below] {6 Digits after D.P};
\end{tikzpicture}
\label{eqn:fixed-point0}
\end{equation}


The following fixed point calculation is $\frac{5}{8} \times \frac{1}{2}$

\begin{equation}
\begin{tikzpicture}[baseline={([yshift=-.5ex]current bounding box.center)}]
    \node[matrix of math nodes] (fixed) {
        \node[fill=red!12](fixed-1-1){0};&\node[fill=red!12]{0};&\node[fill=red!12]{0};&\node[fill=red!12]{0};&
            \node[fill=red!12]{0};&\node[fill=red!12]{1};&\node[fill=red!12]{0};&\node[fill=red!12](fixed-1-8){0};\\
        \empty & \empty & \times & \empty & 0&1&0&\node[fill=red!12](fixed-2-8){1};\\
        \node[fill=blue!12](fixed-3-1){0};&\node[fill=blue!12]{0};&\node[fill=blue!12]{0};&\node[fill=blue!12]{0};&
           \node[fill=blue!12]{0};&\node[fill=blue!12]{1};&\node[fill=blue!12]{0};&\node[fill=blue!12](fixed-3-8){0};\\
        0&0&0&0&0&0&0\\
        0&0&0&1&0&0\\
        0&0&0&0&0\\
        \textbf{\sout{0}}&\textbf{0.}&\textbf{0}&\textbf{1}&\textbf{0}&\textbf{\sout{1}}&\textbf{\sout{0}}&\textbf{\sout{0}}\\
    };
    \draw (fixed-1-8.east) node[right] {$\gets \frac{5}{8}$};
    \draw (fixed-2-8.east) node[right] {$\gets \frac{1}{2}$};
    \draw (fixed-7-1.west) node[left] {Left shift off first value $\to$};
    \draw (fixed-3-1.north west) -- (fixed-3-8.north east); 
    \draw (fixed-7-1.north west) -- (fixed-7-8.north east); 
    \draw[arrow] (fixed-7-2.south) -- ++(0,-1) node[below] {6 Digits after D.P};
\end{tikzpicture}
\label{eqn:fixed-point0}
\end{equation}


\subsubsection{Programming} Programming a DSP can be much more efficient in ASM (Assembly) in terms of
both memory and execution time when compared to C
