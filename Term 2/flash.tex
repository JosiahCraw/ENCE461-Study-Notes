\subsection{Overview}

\begin{table}[H]
\caption{Embedded Memory Comparison Table}
\label{table:mem-overview}
\begin{center}
\begin{adjustbox} {width=\textwidth}
    \begin{tabular} { | m{2cm} | m{2cm} | m{3cm} | m{2cm} | m{2cm} | m{3cm} | m{3cm} | }
        \hline
        \textbf{Type} & \textbf{Volatile} & \textbf{Writeable} & \textbf{Erase Size} & \textbf{Max
        Erase Cycles} & \textbf{Cost (per Byte)} & \textbf{Speed}\\
        \hline
        \textbf{SRAM} & Yes & Yes & Byte & Unlimited & Expensive & Fast\\
        \textbf{DRAM} & Yes & Yes & Byte & Unlimited & Moderate & Moderate\\
        \textbf{Masked ROM} & No & No & N/A & N/A & Inexpensive & Fast\\
        \textbf{PROM} & No & Once, with programmer & N/A & N/A & Moderate & Fast\\
        \textbf{EPROM} & No & Yes, with programmer & Entire Chip & Limited & Moderate to high & Moderate\\
        \textbf{EEPROM} & No & Yes & Byte & Limited & Expensive & Fast Read, Slow Write \& Erase\\
        \textbf{Flash} & No & Yes & Sector & Limited & Moderate & Fast Read, Slow Write \& Erase\\
        \textbf{NVRAM} & No & Yes & Byte & Unlimited & Expensive (SRAM+Battery) & Fast\\
        \hline
    \end{tabular}
\end{adjustbox}
\end{center}
\end{table}


\subsection{NAND Flash}
A NAND memory cell is a modifed transistor with tunneling based programming. The NAND has the
following array structural elements:

\begin{itemize}
    \item String + Bitline are series connected cells
    \item Wordline is across the cells
    \item Block is the smallest unit to be erased
\end{itemize}

The structure of the NAND Flash is shown in Figure \ref{fig:nand}

\begin{figure}[H]
\begin{center}
    \begin{circuitikz}
        \draw (1,-1) node[nmos] (top0) {};
        \draw (top0.D) -- ++(0,0.5) node[vcc] {} node[above, yshift=0.5cm] {B(0)};
        \draw (1,0) node[jump crossing] (00) {};
        \draw (1,-3) node[pmos] (umid0) {};
        \draw (1,-2) node[jump crossing] (01) {};
        \draw (1,-7) node[pmos] (lmid0) {};
        \draw (1,-6) node[jump crossing] (02) {};
        \draw (1,-9) node[nmos] (bottom0) {};
        \draw (1,-8) node[jump crossing] (03) {};
        \draw (bottom0.S) -- ++(0,-0.5) node[ground] {};

        \draw (3,-1) node[nmos] (top1) {};
        \draw (top1.D) -- ++(0,0.5) node[vcc] {} node[above, yshift=0.5cm] {B(1)};
        \draw (3,0) node[jump crossing] (10) {};
        \draw (3,-3) node[pmos] (umid1) {};
        \draw (3,-2) node[jump crossing] (11) {};
        \draw (3,-7) node[pmos] (lmid1) {};
        \draw (3,-6) node[jump crossing] (12) {};
        \draw (3,-9) node[nmos] (bottom1) {};
        \draw (3,-8) node[jump crossing] (13) {};
        \draw (bottom1.S) -- ++(0,-0.5) node[ground] {};
        
        \draw (5,-1) node[nmos] (top2) {};
        \draw (top2.D) -- ++(0,0.5) node[vcc] {} node[above, yshift=0.5cm] {B(2)};
        \draw (5,0) node[jump crossing] (20) {};
        \draw (5,-3) node[pmos] (umid2) {};
        \draw (5,-2) node[jump crossing] (21) {};
        \draw (5,-7) node[pmos] (lmid2) {};
        \draw (5,-6) node[jump crossing] (22) {};
        \draw (5,-9) node[nmos] (bottom2) {};
        \draw (5,-8) node[jump crossing] (23) {};
        \draw (bottom2.S) -- ++(0,-0.5) node[ground] {};
        
        \draw (11,-1) node[nmos] (top3) {};
        \draw (top3.D) -- ++(0,0.5) node[vcc] {} node[above, yshift=0.5cm] {B(i)};
        \draw (11,-3) node[pmos] (umid3) {};
        \draw (11,-7) node[pmos] (lmid3) {};
        \draw (11,-9) node[nmos] (bottom3) {};
        \draw (bottom3.S) -- ++(0,-0.5) node[ground] {};

        % Draw Lines
        \draw (-1,0) node[left] {DSL} -- (00.west) (00.east) -- (10.west) (10.east) -- (20.west)
        (20.east) -- (5.25,0);
        \draw[dashed] (5.25,0) -- (9.5,0);
        \draw (9.5,0) -| (top3.G);
        \draw (top0.G) -| (0,0) node[circ] {};
        \draw (top1.G) -| (2,0) node[circ] {};
        \draw (top2.G) -| (4,0) node[circ] {};

        \draw (-1,-2) node[left] {WL(j)} -- (01.west) (01.east) -- (11.west) (11.east) -- (21.west)
        (21.east) -- (5.25,-2);
        \draw[dashed] (5.25,-2) -- (9.5,-2);
        \draw (9.5,-2) -| (umid3.G);
        \draw (umid0.G) -| (0,-2) node[circ] {};
        \draw (umid1.G) -| (2,-2) node[circ] {};
        \draw (umid2.G) -| (4,-2) node[circ] {};

        \draw (-1,-6) node[left] {WL(0)} -- (02.west) (02.east) -- (12.west) (12.east) -- (22.west)
        (22.east) -- (5.25,-6);
        \draw[dashed] (5.25,-6) -- (9.5,-6);
        \draw (9.5,-6) -| (lmid3.G);
        \draw (lmid0.G) -| (0,-6) node[circ] {};
        \draw (lmid1.G) -| (2,-6) node[circ] {};
        \draw (lmid2.G) -| (4,-6) node[circ] {};
        
        \draw (-1,-8) node[left] {SSL} -- (03.west) (03.east) -- (13.west) (13.east) -- (23.west) 
        (23.east) -- (5.25,-8);
        \draw[dashed] (5.25,-8) -- (9.5,-8);
        \draw (9.5,-8) -| (bottom3.G);
        \draw (bottom0.G) -| (0,-8) node[circ] {};
        \draw (bottom1.G) -| (2,-8) node[circ] {};
        \draw (bottom2.G) -| (4,-8) node[circ] {};

        \draw (top0.S) -- (01.north) (01.south) -- (umid0.S);
        \draw[dashed] (umid0.D) -- (02.north);
        \draw (02.south) -- (lmid0.S) (lmid0.D) -- (03.north) (03.south) -- (bottom0.D); 
        
        \draw (top1.S) -- (11.north) (11.south) -- (umid1.S);
        \draw[dashed] (umid1.D) -- (12.north);
        \draw (12.south) -- (lmid1.S) (lmid1.D) -- (13.north) (13.south) -- (bottom1.D); 

        \draw (top2.S) -- (21.north) (21.south) -- (umid2.S);
        \draw[dashed] (umid2.D) -- (22.north);
        \draw (22.south) -- (lmid2.S) (lmid2.D) -- (23.north) (23.south) -- (bottom2.D); 

        \draw (top3.S) -- (umid3.S);
        \draw[dashed] (umid3.D) -- (11,-6);
        \draw (11,-6) -- (lmid3.S) (lmid3.D) -- (bottom3.D); 


    \end{circuitikz}
\end{center}
\caption{NAND Flash hardware configuration}
\label{fig:nand}
\end{figure}

To read data a address is placed on the word line and the data is read out from the bit line.

\subsubsection{NOR Based Flash}
The difference between NOR Flash and NAND Flash is in NOR flash every memory cell is directly
connected to the bit line. This results in faster read speeds then NAND but slower write and erase
as well as lower density. For these reasons NOR flash is used in MCU's and NAND is used in USB
drives, SD cards and SATA SSDs. The structure of NOR flash is shown in Figure \ref{fig:nor}.

\begin{figure}[H]
\begin{center}
    \begin{circuitikz}
        \draw (0,0) node[pmos, rotate=-90] (mos0) {};
        \draw (2,0) node[pmos, rotate=-90] (mos1) {};
        \draw (4,0) node[pmos, rotate=-90] (mos2) {};
        \draw (6,0) node[pmos, rotate=-90] (mos3) {};

        \draw (mos0.S) -- (mos1.D) node[ground, midway] {};
        \draw (mos2.S) -- (mos3.D) node[ground, midway] {};
        \draw (mos1.S) -- (mos2.D) node[midway] (helper0) {};
    
        \draw (mos0.G) node[above] {WL(0)};
        \draw (mos1.G) node[above] {WL(1)};
        \draw (mos2.G) node[above] {WL(2)};
        \draw (mos3.G) node[above] {WL(3)};

        \draw (-2,2) node[left] {Bit Line} -- (8,2);
        \draw (mos0.D) -- (mos0.D |- -1,2);
        \draw (mos3.S) -- (mos3.S |- -1,2);
        \draw (3, |- mos0.D) -- (3,2);
    \end{circuitikz}
\end{center}
\caption{NOR Flash Structure}
\label{fig:nor}
\end{figure}


\subsection{Dynamic Memory Allocation}
Dynamic memory allocation using commands like `\textit{malloc()}' and `\textit{free()}' allocate and
deallocate memory from the heap. Dynamic memory presents issues with transparency as local
variables defined by `\textit{malloc()}' are difficult to maintain. `\textit{malloc()}' can also
make embedded software non deterministic under a Real time operating system (RTOS) as memory
allocation time can be slow.

\subsection{Heap Fragmentation}
Heap fragmentation is when the through the process of allocation and freeing the heap gets split
into parts by allocated blocks. These blocks prevent allocation as no contiguous blocks exist due to
the segmented smaller allocated blocks.

\subsection{Global Variables}
Global variables are used to allow access between foreground and background tasks. However, this
makes them \textit{volatile} as they may be changed unexpectedly therefore they cannot be stored in
registers. These global variable must then be stored in RAM, usually SRAM throughout the code run,
this enables faster access to the variable. There are risks with the use of global variables, such
as race conditions some of these risks can be mitigated by using atomic operations (One clock cycle
to complete/cannot be interrupted)

\subsection{Memory Overlays}
Memory overlays are the where different sections of code can be read into internal memory,
overwriting another. This is used to accommodate limited SRAM. To use memory overlays the code/data
must be structured in such a way that not all of the data is required in memory at the same time eg.
code is segmentable or large datasets are not required all at once. Examples of data would be FFT,
Digital filtering and image data. These are represented in C as a `union' an example of an overlay
tree is shown in Figure \ref{fig:overlay}.

\begin{figure}[H]
\begin{center}
\begin{adjustbox} {width=\textwidth}
    \begin{tikzpicture}
        \node[boxes] (root) {Root};
        \node[boxes, below of=root, xshift=-6cm, yshift=-1cm] (a) {A};
        \node[boxes, below of=root, yshift=-1cm] (b) {B};
        \node[boxes, below of=root, xshift=6cm, yshift=-1cm] (c) {C};
        \node[boxes, below of=a, xshift=-2cm, yshift=-1cm] (d) {D};
        \node[boxes, below of=a, xshift=2cm, yshift=-1cm] (e) {E};
        \node[boxes, below of=b, yshift=-1cm] (f) {F};
        \node[boxes, below of=c, yshift=-1cm] (g) {G};

        \draw (root.north) node[above] {2kB};
        \draw (a.west) node[left] {4kB};
        \draw (b.west) node[left] {6kB};
        \draw (c.west) node[left] {8kB};

        \draw (d.west) node[left] {6kB};
        \draw (e.west) node[left] {8kB};
        \draw (f.west) node[left] {2kB};
        \draw (g.west) node[left] {4kB};

        \draw[arrow] (root.south) -- (a.north);
        \draw[arrow] (root.south) -- (b.north);
        \draw[arrow] (root.south) -- (c.north);
        
        \draw[arrow] (a.south) -- (d.north);
        \draw[arrow] (a.south) -- (e.north);
        \draw[arrow] (b.south) -- (f.north);
        \draw[arrow] (c.south) -- (g.north);

    \end{tikzpicture}
\end{adjustbox}
\end{center}
\caption{Memory Overlay Tree example}
\label{fig:overlay}
\end{figure}


\subsection{Erasing and Writing}
For NAND Flash erasing has to occur by the page. Whereas NOR can erase individual bits. To perform
an erase a large reverse polarity voltage is applied to the floating gate. NAND has to write 
an entire page at a time. NOR's ability to erase/write individual bits increases its random 
read/write speed making it more suitable for RAM (Random Access Memory).

\subsection{Reading}
To read the NAND Flash applies a voltage to the word line and the bit line the voltage is then read
from the bit line to determine the bit state. NOR can read faster than NAND making it suitable for
high speed applications however, it has a lower density then NAND so is not used for large scale
storage such as USB drives.
