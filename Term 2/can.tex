Control Area Network (CAN) is a noise resilient, asynchronous half-duplex serial protocol. The
protocol uses a broadcast bus topology with no specific device/chip address. CAN also has message
identifiers and message priority as well as type. Messages can also be processed by multiple nodes
(\textit{functional addressing})\\

\noindent The two wire structure supports 1Mbps for distances 40m, 120kbps possible over 500m and 
5kbps over 10km. CAN-FD (\textit{flexible data rate}) increased the speed by a factor of 5.\\

\noindent CAN uses multi-master/controller system with carrier sense multiple access with collision
detection (CSMA/CD). CAN also uses arbitration on message priority (AMP), this is a non-destructive
form of arbitration with is important for hard deadline, real-time and distributed systems. The
priority order is the lower the number the higher the priority ie. 0 is the highest priority.\\

\noindent CAN uses non-return to zero (NRZ) line coding this causes difficulty in clock
synchronisation for long strings of `1' and `0'. This means that CAN requires falling or rising
edges for every 5 bits, so after five bits of the same level, a 6\textsuperscript{th} inverse bit is
added to the transmission. These bits are then removed at the receivers.

\begin{figure}[H]
\begin{center}
    \begin{tikzpicture}
        \node[wboxes, line width=2pt] (micro) {Microcontroller};
        \node[wboxes, below of=micro, yshift=-1cm] (control) {CAN Controller};
        \node[wboxes, below of=control, yshift=-1cm] (tran){CAN Transceiver};

        \draw[line width=2pt] (-3,-5) node[left] {CAN Bus} -- (3,-5);
        \draw[line width=2pt] (-3,-6) node[left] {CAN Bus} -- (3,-6);
        \draw[darrow] (-1,-5) -- (-1,-4.5);
        \draw[darrow] (1,-6) -- (1,-4.5);
        \draw[darrow] (tran) -- (control);
        \draw[darrow] (control) -- (micro);

    \end{tikzpicture}
\end{center}
\caption{CAN Block Diagram}
\label{fig:can=block}
\end{figure}

\noindent The two wires used in CAN are a differential pair, this increases noise resilience which
is important in the automotive sector where this protocol was designed. This bus also still works if
the bus is broken of shorted to either power or ground. To represent a `1' bit the pair is
\textit{recessive}, this is where $\textrm{CAN}_H \approx \textrm{CAN}_L \approx V_{cc}/2$ to
represent a `0' bit the pair is \textit{dominant} this is where $\textrm{CAN}_H > \textrm{CAN}_L$
this behaviour is shown in Figure \ref{fig:can-wave}.

\begin{figure}[H]
\begin{center}
    \begin{tikzpicture}
    \begin{axis}[
        width=\textwidth,
        axis x line=middle,
        axis y line=left,
        xmajorgrids=true,
        ymajorgrids=true,
        grid style=dashed,
        xtick={0,10,...,120},
        ytick={\empty},
        extra y ticks={0,1,2,3,5},
        extra y tick labels={$0V$, $2.5V$, $5V$, $0V$, $5V$},
        xmin=-10,
        xmax=125,
        ymin=-0.5,
        ymax=6,
        samples=1000,
        title=CAN Differential Signalling,
        xticklabels=\empty,
    ]

    \addplot[black] coordinates
    {(0,3) (0,5) (10,5) (10,3) (20,3) (20,5) (40,5) (40,3) (70,3) (70,5) (80,5) (80,3) (90,3) (90,5)
    (100,5) (100,3) (110,3)};

    \addplot[blue, name path=canh] coordinates
    {(0,1) (10,1) (10,2) (20,2) (20,1) (40,1) (40,2) (70,2) (70,1) (80,1) (80,2) (90,2) (90,1)
    (100,1) (100,2) (110,2)};

    \addplot[red, name path=canl] coordinates
    {(0,1) (10,1) (10,0) (20,0) (20,1) (40,1) (40,0) (70,0) (70,1) (80,1) (80,0) (90,0) (90,1)
    (100,1) (100,0) (110,0)};

    \path[name path=middle] (axis cs:0,1) -- (axis cs:110,1);
    \addplot[
        color=red,
        fill=red,
        fill opacity=0.1,
    ] fill between[
        of=middle and canl,
        soft clip={domain=0:110},
    ];
    
    \addplot[
        color=blue,
        fill=blue,
        fill opacity=0.1,
    ] fill between[
        of=middle and canh,
        soft clip={domain=0:110},
    ];

    % Draw Bits
    \draw (axis cs:5,4) node[] {1}
        (axis cs:15,4) node[] {0}
        (axis cs:25,4) node[] {1}
        (axis cs:35,4) node[] {1}
        (axis cs:45,4) node[] {0}
        (axis cs:55,4) node[] {0}
        (axis cs:65,4) node[] {0}
        (axis cs:75,4) node[] {1}
        (axis cs:85,4) node[] {0}
        (axis cs:95,4) node[] {1}
        (axis cs:105,4) node[] {0};
    

    % Draw Labels
    \draw (axis cs:55,1.5) node[] {CAN$_H$};
    \draw (axis cs:55,0.5) node[] {CAN$_L$};
    \draw (axis cs:55,2.75) node[] {NRZ Data from CAN Controller};
    \draw (axis cs:55,2.25) node[] {CAN bus Differential Signal};
    
    \end{axis}
    \end{tikzpicture}
\end{center}
\caption{CAN Differential Signalling}
\label{fig:can-wave}
\end{figure}


The idle state of CSMA CAN is recessive (`1'). To avoid collisions CAN monitors bit transmissions.

\subsection{CAN Message Format}
The following figure describes the packet format of a CAN message.

\begin{figure}[H]
\begin{center}
    \begin{tikzpicture}
        \node[boxes, minimum height=2cm, minimum width=\textwidth] at (7.5,0) (box) {};

        % Fill the boxes first to stop overlaying text
        \draw[fill=gray, fill opacity=0.08] (4,1) -- (5,1) -- (5,-1) -- (4,-1) -- (4,1);
        \draw[fill=gray, fill opacity=0.08] (5,1) -- (6,1) -- (6,-1) -- (5,-1) -- (5,1);
        \draw[fill=gray, fill opacity=0.08] (6,1) -- (7,1) -- (7,-1) -- (6,-1) -- (6,1);
        \draw[fill=blue, fill opacity=0.08] (7,1) -- (11,1) -- (11,-1) -- (7,-1) -- (7,1);
        \draw[fill=gray, fill opacity=0.08] (11,1) -- (12,1) -- (12,-1) -- (11,-1) -- (11,1);
        \draw[fill=blue, fill opacity=0.08] (12,1) -- (13,1) -- (13,-1) -- (12,-1) -- (12,1);


        \draw (1,1) -- (1,-1);
        \draw (0.5,0) node[rotate=90] {\textbf{SOF}};

        \draw (3,1) -- (3,-1);
        \draw (2,0) node[] {11-bit ID};

        \draw (4,1) -- (4,-1);
        \draw (3.5,0) node[rotate=90] {\textbf{RTR}};

        \draw (5,1) -- (5,-1);
        \draw (4.5,0) node[rotate=90] {\textbf{IDE}};

        \draw (6,1) -- (6,-1);
        \draw (5.5,0) node[] {\textbf{r0}};

        \draw (7,1) -- (7,-1);
        \draw (6.5,0) node[rotate=90] {\textbf{DLC}};

        \draw (11,1) -- (11,-1);
        \draw (9,0) node[] {0-8 Data Bytes};

        \draw (12,1) -- (12,-1);
        \draw (11.5,0) node[rotate=90] {\textbf{CRC}};

        \draw (13,1) -- (13,-1);
        \draw (12.5,0) node[rotate=90] {\textbf{ACK}};

        \draw (14,1) -- (14,-1);
        \draw (13.5,0) node[rotate=90] {\textbf{EOF}};

        \draw (14.5,0) node[rotate=90] {\textbf{IFS}};

        % Draw Key
        \draw (0,-1.5) node[right] {\textbf{SOF}: Start of frame}
            ++(0,-0.5) node[right] {\textbf{ID}: Priority and type of the message. The lower the
            binary value, the higher the priority}

            ++(0,-0.5) node[right] {\textbf{RTR}: Remote transmission request bit is zero when
            information is requested from another node}

            ++(0,-0.5) node[right] {\textbf{IDE}: Identifier Extension Bit}
            ++(0,-0.5) node[right] {\textbf{r0}: Reserved Bit}
            ++(0,-0.5) node[right] {\textbf{DCL}: Four Bit Data Length}
            ++(0,-0.5) node[right] {\textbf{Data}: Up to 8 bytes of data}
            ++(0,-0.5) node[right] {\textbf{CRC}: Cyclic Redundancy Check}
            ++(0,-0.5) node[right] {\textbf{ACK}: Acknowledge from every node that receives the
            message correctly}

            ++(0,-0.5) node[right] {\textbf{EOF}: End of File (7-bit)}
            ++(0,-0.5) node[right] {\textbf{IFS}: Inter-frame space for intermission};
    \end{tikzpicture}
\end{center}
\caption{CAN Packet Format}
\label{fig:can-packet}
\end{figure}


\subsection{CRC}
\paragraph{Parity Checks} Parity checks can be used to verify a transmission, using a \textbf{single
bit} XOR gates are used to determine the value of the parity bit. Essentially a parity bit of `1'
means that there is an odd number of `1' bits and `0' means that there is an even number of `1' bits.

\noindent The method of a parity check is given by (\ref{eqn:parity}) where the bit sequence $x$ is
`0010 0101'

\begin{equation}
\begin{aligned}
    x &= 00100101\\
    \textrm{CRC} &= x_7 \oplus x_6 \oplus x_5 \oplus x_4 \oplus x_3 \oplus x_2 \oplus x_1 \oplus
    x_0\\
    &=0 \oplus 0 \oplus 1 \oplus 0 \oplus 0 \oplus 1 \oplus 0 \oplus 1\\
    \textrm{CRC} &= 1
\end{aligned}
\label{eqn:parity}
\end{equation}

\subsubsection{CRC}
Cyclic Redundancy Check (CRC) is a $n$-bit checksum method which detects data inconsistency using
polynomial division. This uses a \textit{Dividend} which is the input data, and \textit{Divisor}
which is a given bit sequence also known as a \textit{generator polynomial} which is defined as
(\ref{eqn:generator}). The predefined generator polynomial contains $n+1$ bits.

\begin{equation}
    G=\sum_{k=0}^{k=n+1} x^k
    \label{eqn:generator}
\end{equation}

To preform a CRC:

\begin{enumerate}
\item Append the input with $n$ `0'
\item Align the 1\textsuperscript{st} `1' of the divisor with the 1\textsuperscript{st} `1' of the
dividend
\item Stop when the divisor zeros out each bit of input data
\item Return the last $n$ bits of the remainder
\end{enumerate}

A CRC-3 (3-bit CRC) check example is shown in (\ref{eqn:crc}) this shows the process of completing
the first step, with each step repeating the same way.

\begin{equation}
\begin{tikzpicture}[baseline={([yshift=-.5ex]current bounding box.center)}]

\node[matrix of math nodes] (div) {
    \empty & \empty & \empty & 1 & 1 & 1 & 1\\
    1 & \node[fill=red!10]{1}; & \node[fill=blue!10]{0}; & \node[fill=red!10]{1}; &
    \node[fill=blue!10]{0}; & 0 & 0\\
    1 & \node[fill=red!10]{0}; & \node[fill=blue!10]{1}; & \node[fill=red!10]{1};\\
    \empty & \node[fill=red!10]{\textbf{1}}; & \node[fill=blue!16]{\textbf{1}}; &
    \node[fill=red!10]{\textbf{0}}; & \node[fill=blue!10]{\textbf{0}};\\
    \empty & 1 & 0 & 1 & 1\\
    \empty & \empty & 1 & 1 & 1 & 0\\
    \empty & \empty & 1 & 0 & 1 & 1\\
    \empty & \empty & \empty & 1 & 0 & 1 & 1\\
    \empty & \empty & \empty & 1 & 0 & 1 & 1\\
    \empty & \empty & \empty & \empty & 0 & 0 & 1\\
};

\draw (div-3-1.south west) -- (div-2-7.south east |- div-3-1.south); 
\draw (div-5-2.south west) -- (div-2-7.south east |- div-5-2.south); 
\draw (div-7-3.south west) -- (div-2-7.south east |- div-7-3.south); 
\draw (div-9-4.south west) -- (div-2-7.south east |- div-9-4.south); 

\draw (div-2-7.east) node[right] {$\gets$ Input (Padded with $N$ (3) Zeros)};
\draw (div-10-7.east) node[right] {$\gets$ Remainder};

\draw (div-2-1) node[left, xshift=-0.2cm] {Generator Polynomial $\to$\textbf{1011}};
\draw (div-2-1.north west) -- (div-2-7.north east);
\draw (div-2-1.south west) -- (div-2-1.north west);
\end{tikzpicture}
\label{eqn:crc}
\end{equation}


\textbf{NOTE}: CAN uses CRC-15 so the generator polynomial is 16bit long and the remainder is 15bit
long.


