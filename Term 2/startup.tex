\subsection{Memory Sections}
A program for an embedded system generally consists of more than one source file.
After compilation these files are organised into different memory sections:

\begin{itemize}
    \item \textbf{.text} - Code and instructions
    \item \textbf{.rodata} - Constraints {\small (eg. pi cosine/sine tables, FIR filter coefficients etc.)}
    \item \textbf{.data} - Initialised Data {\small (Global, static and local variables with initial values)}
    \item \textbf{.bss} - Uninitialised Data {\small {Global, static and local variables that are uninitialised or initialised to 0}}
\end{itemize}

The \textbf{.text}, \textbf{.rodata} and \textbf{.data} segments all are held within flash memory as they casn be fetched when required and
do not need to be held in SRAM.\\

The \textbf{.data} and \textbf{.bss} are held within SRAM as they may change with ht program therefore SRAM is the most suitable.\\

\textbf{NOTE:} The \textbf{.data} segment can be held within both, this is because initialised variables may be changed throughout
the course of a program run and therefore maybe moved to SRAM.\\

As shown in Figure \ref{fig:mem-stack} the heap and the stack typically grow toward each other.

\begin{figure}[H]
    \begin{center}
        \begin{tikzpicture}
            \node(text) [boxes] {\textbf{text}};
            \node(data) [boxes, below of=text] {\textbf{data}\\{\tiny Initialised Data}};
            \node(bss) [boxes, below of=data] {\textbf{bss}\\{\tiny Uninitalised Data}};
            \node(heap) [boxes, below of=bss, minimum height=3cm, yshift=-1cm] {\textbf{heap}};
            \node(stack) [boxes, below of=heap, minimum height=3cm, yshift=-2cm] {\textbf{stack}};
            \draw[dasharrow] (0, -4.25) -- (stack);
            \draw[dasharrow] (0, -6.75) -- (heap);
        \end{tikzpicture}
    \end{center}
    \caption{Memory Stack}
    \label{fig:mem-stack}
\end{figure}

\textbf{Heap}, the heap contains dynamic memory allocations, this area is what is provided when
malloc(), calloc(), realloc() and free() are called in C. The heap starts at the bottom of the bss section 
and grows down toward the stack.\\

\textbf{Stack}, the stack has a Last in First out (LIFO) structure and starts at the end of SRAM and grows
toward smaller address numbers.

\subsection{Linking}
Linking is the mapping of certain sections of physical memory to virtual memory space.

\begin{itemize}
    \item Code and Constraints (.text and .rodata) $\to$ FLASH {\tiny (Read Only)}
    \item Uninitialised variables (.bss) $\to$ SRAM {\tiny (Read Write)}
    \item Initialised variables (.data) $\to$ FLASH and SRAM {\tiny (copy data before runtime for both read and write)}
    \item Interrupt vector table $\to$ start of FLASH
\end{itemize}

\subsubsection{Memory Map}

\begin{figure}[H]
    \begin{center}
        \begin{tikzpicture}
            \node(code) [wboxes] {Code};
            \node(sram) [wboxes, below of=code, fill=blue!8] {SRAM};
            \node(undefined) [wboxes, below of=sram, fill=gray!8] {Undefined};
            \node(bitband) [wboxes, below of=undefined, fill=blue!8] {32MB bit band alias};
            \node(periph) [wboxes, below of=bitband] {Peripherals};
            \node(undef1) [wboxes, below of=periph] {Undefined}; 
            \node(externalsram) [wboxes, below of=undef1] {External SRAM};
            \node(reserved) [wboxes, below of=externalsram, fill=gray!20] {Reserved};
            \node(system) [wboxes, below of=reserved] {System};

            % Address stuff
            \draw node[left] at (code.north west) {\tiny 0x00000000};
            \draw node[left] at (sram.north west) {\tiny 0x20000000};
            \draw node[left] at (undefined.north west) {\tiny 0x20400000};
            \draw node[left] at (bitband.north west) {\tiny 0x22000000};
            \draw node[left] at (undef1.north west) {\tiny 0x24000000};
            \draw node[left] at (periph.north west) {\tiny 0x40000000};
            \draw node[left] at (externalsram.north west) {\tiny 0x60000000};
            \draw node[left] at (reserved.north west) {\tiny 0xA0000000};
            \draw node[left] at (system.north west) {\tiny 0xE0000000};
            \draw node[left] at (system.south west) {\tiny 0xFFFFFFFF};

            \draw node[above] at (code.north) {\small Address Memory Space};
            
            % Draw Code sub memory
            \node(bootmem) [boxes, left of=code, xshift=-5cm] {\small Boot Memory};
            \node(intflash) [boxes, below of=bootmem] {\small Internal Flash};
            \node(introm) [boxes, below of=intflash] {\small Internal ROM};
            \node(reserv) [boxes, below of=introm, fill=gray!20] {\small Reserved};

            % Draw Labels
            \draw node[left] at (bootmem.north west) {\tiny 0x00000000};
            \draw node[left] at (intflash.north west) {\tiny 0x00400000};
            \draw node[left] at (introm.north west) {\tiny 0x00800000};
            \draw node[left] at (reserv.north west) {\tiny 0x00C00000};
            \draw node[left] at (reserv.south west) {\tiny 0x1FFFFFFF};

            % Draw Lines
            \draw[dashed] (code.north west) -- (bootmem.north east);
            \draw[dashed] (code.south west) -- (reserv.south east);

            % Draw external RAM sub memory
            \node(smc0) [boxes, below of=reserv, yshift=-2cm] {\small SMC Chip Select 0};
            \node(smc1) [boxes, below of=smc0] {\small SMC Chip Select 1};
            \node(smc2) [boxes, below of=smc1] {\small SMC Chip Select 2};
            \node(smc3) [boxes, below of=smc2] {\small SMC Chip Select 3};
            \node(res) [boxes, below of=smc3, fill=gray!20] {\small Reserved};

            % Draw Labels
            \draw node[left] at (smc0.north west) {\tiny 0x60000000};
            \draw node[left] at (smc1.north west) {\tiny 0x61000000};
            \draw node[left] at (smc2.north west) {\tiny 0x62000000};
            \draw node[left] at (smc3.north west) {\tiny 0x63000000};
            \draw node[left] at (res.north west) {\tiny 0x64000000};
            \draw node[left] at (res.south west) {\tiny 0x9FFFFFFF};

            % Draw Lines
            \draw[dashed] (externalsram.north west) -- (smc0.north east);
            \draw[dashed] (externalsram.south west) -- (res.south east);

            
        \end{tikzpicture}
    \end{center}
    \caption{Memory Mappings}
    \label{fig:mem-map}
\end{figure}


From Figure \ref{fig:mem-map}:

\begin{itemize}
    \item The system always boots at 0x00000000
    \item Internal ROM from 0x080000000 contains:
        \begin{itemize}
            \item SAM boot assistant
            \item In-application programming (IAP) routines
            \item Fast FLASH programming interface (FFPI)
            \item Easy way for programming \todo{in-situ} the on chip FLASH memory via serial communication UART/USB SAM-BA GUI provided
        \end{itemize}
    \item Internal Flash, first address contains the interrupt register
    \item Boot using SAM-BAon the ROM (\textit{default})
        \begin{itemize}
            \item General Purpose NVM (GPNVM) bit=0
        \end{itemize}
    \item Boot using SAM-BA on FLASH
        \begin{itemize}
            \item General-purpose NVM (GPNVM) bit=1
        \end{itemize}
    \item Enhanced embedded FLASH controller (EEFC) user interface: set GPNVM bit
\end{itemize}

\subsection{Linker Script}
\begin{itemize}
    \item MEMORY Figure \ref{fig:linker-mem}
    \item SECTIONS Figure \ref{fig:linker-sections}
    \item .text Figure \ref{fig:linker-text}
    \item .data Figure \ref{fig:linker-data}
\end{itemize}
Firstly, this linker script specifies the memory space according to the previously defined memory map.

\begin{figure}[H]
    \begin{center}
    \begin{lstlisting}[language=C]
    MEMORY
    {
        FLASH (rx) : ORIGIN = 0x00400000, LENGTH = 1M
        SRAM (rwx) : ORIGIN = 0x20000000, LENGTH = 128K
    }
    \end{lstlisting}
    \end{center}
    \caption{Linker Script memory declaration}
    \label{fig:linker-mem}
\end{figure}

Where, FLASH is readable and executable `(rx)' and SRAM is readable, writable and executable `(rwx)'\\

Next, physical memory addresses are specified for sections as shown in Figure \ref{fig:linker-sections}

\begin{figure}[H]
    \begin{center}
    \begin{lstlisting}[language=C]
    SECTIONS
    {
        . = 0x0000000; // `.' is the location counter
        /* ... */
    }
    \end{lstlisting}
    \end{center}
    \caption{Linker Script Section declaration}
    \label{fig:linker-sections}
\end{figure}

All of the \textbf{.text} sections from all of the object files are combined into a single \textbf{.text} section with
the following script.

\begin{figure}[H]
    \begin{center}
    \begin{lstlisting}[language=C]
    .text :
    {
        /* .text section starts with an address with the last two bits = `00' */
        . = ALIGN(4); 
        /* Concatenate all .text sections */
        *(.text)
    } >FLASH // Put the .text section in FLASH
    \end{lstlisting}
    \end{center}
    \caption{Linker Script Section declaration}
    \label{fig:linker-text}
\end{figure}

The \textbf{.data} sections are loaded into FLASH and SRAM at startup

\begin{figure}[H]
    \begin{center}
    \begin{lstlisting}[language=C]
    .data : AT (__text_end__)
    {
        __data_start__ = .;
        *(.data)
        __data_end__ = .;
        __data_load__ = LOADADDR (.data); // Returns load memory address of .data section
    } >SRAM // Put the .data section in SRAM

    /*
        AT is the starting load memory address of the .data section in FLASH
        __data_start__, __data_end__ and _  __data_load__ are SRAM addresses
    */
    \end{lstlisting}
    \end{center}
    \caption{Linker Script Section declaration}
    \label{fig:linker-data}
\end{figure}

\subsection{Boot from RESET}
The SAM4S clock initally runs off of the internal 4MHz RC oscillator\\
On RESET the the SAM4S loads the program counter (PC) with the address of the RESET interrupt service routine (ISR)\\

The following code Figure \ref{fig:boot-reset} describes the reset handler:
\begin{figure}[H]
    \begin{center}
    \begin{lstlisting}[language=C]
    void _reset_handler(void) {
        /* Init the preinitialised variables in .data */
        for (char *src = & __data_load__, char *dst = &__data_start__; dst < &__data_end__) {
            *dst++ = *src++;
        }

        /* Zero all remaining uninitialised global variables in .bss */
        for (dst = &__bss_start__; dst < &__bss_end__; ) {
           *dst++ = 0;
        }

        /* 
        Remap the exception table into SRAM to allow dynamic 
        allocation. This register is zero on reset 
        */
        SCB->VTOR = (uint32_t)&exception_table & SCB_VTOR_TBLOFF_Msk;

        /* Set up clockes, etc. */
        mcu_init();

        main();

        /* Hang on program termination */
        while (1) { continue; }
    }
    \end{lstlisting}
    \end{center}
    \caption{Reset Handler code snippet}
    \label{fig:boot-reset}
\end{figure}


\begin{figure}[H]
    \begin{center}
    \begin{lstlisting}[language=C]
    void mcu_init(void) {
        /* Disable all interrupts */
        for (int i = 0; i < 8; i++) {
            /* 
            ICER0 - ICER7: 32 bit vector indicating the 
            enabled/disabled interrupts 
            */
            NVIC->ICER[i] = ~0;
        }

        mcu_reset_enable();

        /* Reduce the number of wait states for the flash memory */
        mcu_flash_init();

        mcu_watchdog_disable();
        mcu_clock_init();
        irq_global_enable();
    }
    \end{lstlisting}
    \end{center}
    \caption{MCU initialisation code snippet}
    \label{fig:boot-mcu-init}
\end{figure}

