Delays are used frequently in MCU programming however it is difficult to ensure consistantcy and
accuracy for long delays as interrupts can cause accuracy issues for timers making them unrelaible
for both long time periods and in complicated programs. There will always be a trade off between
time resolution and delay range.

\subsection{Time Resolution and Delay Range}
Delays can have any length, depending on the clock speed of the unit driving it. The
\textit{Granularity} is the measure of a chosen smallest time unit of a delay and the \textit{Range}
is a function the number of core bits/maximum count of a number in an MCU. This range effectively
dictates the maximum time a delay can be, taking into account the granularity and the maximum count
of an MCU.

\begin{table}[H]
\caption{Time resolution and delay range table}
\label{table:time-reso}
\begin{center}
    \begin{tabular} {| m{3cm} | m{3cm} | m{3cm} | m{3cm} | }
        \hline
        Granularity & Range (8-bit) & Range (16-bit) & Range (32-bit)\\
        \hline
        $1\mu \textrm{s}$ & $255\mu \textrm{s}$ & $66\textrm{ms}$ & $1\textrm{min}$\\
        $1\textrm{ms}$ & $255\textrm{ms}$ & $66\textrm{s}$ & $50\textrm{days}$\\
        $1\textrm{s}$ & $4.3\textrm{min}$ & $18\textrm{hr}$ & $136\textrm{yr}$\\
        \hline
    \end{tabular}
\end{center}
\end{table}


To calculate delay length (\ref{eqn:delay}) is used where, $\tau_0$ is the setup delay, $\Delta_tau$
is the timing resolution and $n$ is the number of ticks creating the delay length. The fundimental
tradeoff between delay range and timing resolution under given bit width is shown in
(\ref{eqn:fund-trade-off})

\begin{equation}
    \tau (n) = \tau_0 + n \Delta_\tau
    \label{eqn:delay}
\end{equation}

\begin{equation}
    \tau_{\textrm{max}}=\tau_0 + (2^{M}-1) \Delta_\tau
    \label{eqn:fund-trade-off}
\end{equation}

\subsection{CPU Based Delay}
CPU Based delay is where the CPU clock rates and architecture determine the granularity and delay
range.

\begin{table}[H]
\caption{CPU Clock delay table}
\begin{center}
    \begin{tabular} { | m{3cm} | m{3cm} | m{3cm} |}
        \hline
        CPU Speed & Delay ($1\mu \textrm{s}$) & Delay ($10\mu \textrm{s}$)\\
        \hline
        2MHz & 2 cycles & 20 cycles\\
        20MHz & 20 cycles & 200 cycles\\
        200MHz & 200 cycles & 2000 cycles\\
        2GHz & 2000 cycles & 20000 cycles\\
        \hline
    \end{tabular}
\end{center}
\end{table}


\subsection{Busy Wait/Delay Loops}
Delay Loops are dependent on CPU clock frequency for time resolution as well as architecture for the
delay range. These delay loops run as background tasks which means that the use of them is very
resource intensive on the CPU. This also means that it more accurate for shorter periods.\\

\noindent Busy wait is when a peripheral is `\textit{idle}' initially and waits to either receive data if it is
an input device or receive instructions from the MCU if it is an output device. After data is
received the peripheral starts in `\textit{busy}' state and after completion it moves to a
`\textit{done}' state where it waits for the next instruction. 

\subsection{Timeouts}
Timeouts prevent an embedded system from entering a non-responsive state while waiting for a
particular external event. This is achieved by allowing the system to exit the non-responsive state
after a set amount of time waiting for an external trigger. Eg A temperature sensor reading every 1s
in a case where the sensor is unresponsive and the MCU waits forever for it to respond. To fix this
the MCU may exit the waiting state after a set period to stop issues with the sensor.
