\begin{itemize}
    \item \textbf{Debugging} Debugging allows the programmer to verify the correctness of a program. The debugger runs the program
    and is able to detect the current state of the program including variable values and register states.

    \item \textbf{Profiling} Profiling collects a time record of the program execution 
\end{itemize}

\textbf{Debugger Types:}
\begin{itemize}
    \item \nameref{section:JTAG}
    \item \nameref{section:SWD}
    \item \nameref{section:OpenOCD}
    \item \nameref{section:GDB}
\end{itemize}

\subsection{JTAG}
\label{section:JTAG}
Joint Test Action Group (JTAG) is a specialized serial bus developed in the 1980s it is defined in the standard IEEE 1491 in 1990.
The main use of this bus is to program and debug microprocessors and other programmable logic devices. This standard provides a test access
port and boundary scan architecture for integrated circuits. \textbf{Boundary Scan Architecture} allows values on pins to be read from and written
to without the use of external tools such as oscilloscopes and signal generators.

\begin{figure}[H]
\begin{adjustbox} {max width=1\textwidth, center}
\centering   
\begin{circuitikz}
    \ctikzset{multipoles/thickness=3}
    \ctikzset{multipoles/dipchip/width=2}

    % U0
    \draw (0, 0) node[dipchip, num pins=8, hide numbers, no topmark,
        external pins width=0](U0) {U0};

    \node [right, font=\tiny] at (U0.bpin 1) {TMS};
    \node [right, font=\tiny] at (U0.bpin 2) {TCK};
    \draw (U0.bpin 2) ++(0, 0.1) -- ++(0.1, -0.1);
    \draw (U0.bpin 2) ++(0, -0.1) -- ++(0.1, 0.1);
    \node [right, font=\tiny] at (U0.bpin 4) {TDI};

    \node [left, font=\tiny] at (U0.bpin 5) {TDO};

    % U1
    \draw (5, 0) node[dipchip, num pins=8, hide numbers, no topmark,
    external pins width=0](U1) {U1};

    \node [right, font=\tiny] at (U1.bpin 1) {TMS};
    \node [right, font=\tiny] at (U1.bpin 2) {TCK};
    \draw (U1.bpin 2) ++(0, 0.1) -- ++(0.1, -0.1);
    \draw (U1.bpin 2) ++(0, -0.1) -- ++(0.1, 0.1);
    \node [right, font=\tiny] at (U1.bpin 4) {TDI};

    \node [left, font=\tiny] at (U1.bpin 5) {TDO};

    % U2
    \draw (10, 0) node[dipchip, num pins=8, hide numbers, no topmark,
    external pins width=0](U2) {U2};

    \node [right, font=\tiny] at (U2.bpin 1) {TMS};
    \node [right, font=\tiny] at (U2.bpin 2) {TCK};
    \draw (U2.bpin 2) ++(0, 0.1) -- ++(0.1, -0.1);
    \draw (U2.bpin 2) ++(0, -0.1) -- ++(0.1, 0.1);
    \node [right, font=\tiny] at (U2.bpin 4) {TDI};

    \node [left, font=\tiny] at (U2.bpin 5) {TDO};

    % Drawing the connections
    \node [left] at (-4, 3) {TMS};
    \node [left] at (-4, 2) {TCK};
    \node [left] at (-4, 52 |- U0.bpin 4) {TDI};
    \node [left] at (-4, -2) {TDO};


    \node[circ] at (-2, 3) {};
    \node[circ] at (-2.5, 2) {};
    \node[jump crossing] at (-2, 2) (cr0) {};
    \draw (-4, 3) -- (-2, 3);
    \draw (-4, 2) -- (-2.5, 2);
    \draw (-2.5,  2) -- (cr0.west);
    \draw (-2, 3) |- (U0.bpin 1);
    \draw (-2.5, 2) |- (U0.bpin 2);
    \draw (-4, 52 |- U0.bpin 4) -- (U0.bpin 4);
    \draw (U0.bpin 5) -- (U1.bpin 4);

    \node[circ] at (3, 3) {};
    \node[circ] at (2.5, 2) {};
    \node[jump crossing] at (3, 2) (cr1) {};
    \draw (-2, 3) -- (3, 3);
    \draw (cr0.east) -- (cr1.west);
    \draw (3, 3) |- (U1.bpin 1);
    \draw (2.5, 2) |- (U1.bpin 2);
    \draw (U1.bpin 5) -- (U2.bpin 4);

    \draw (3, 3) -- (8, 3);
    \draw (cr1.east) -- (7.5, 2);
    \draw (8, 3) |- (U2.bpin 1);
    \draw (7.5, 2) |- (U2.bpin 2);
    \draw (U2.bpin 5) -| (12, -2);
    \draw (12, -2) -- (-4, -2);

\end{circuitikz}
\end{adjustbox}
\caption{Daisy Chained JTAG Ports}
\label{fig:jtag}
\end{figure}


JTAG contains the following possible signals:
\begin{enumerate}
    \item \textbf{TCK} - Test Clock for clocking test data entering on TDI and TDO
    \item \textbf{TMS} - Test mode state: frame sync and JTAG mode switch
    \item \textbf{TDI} - Test data in
    \item \textbf{TDO} - Test data out
    \item \textbf{TRST} - Test reset (\textit{optional}, 4+1 interface) to reset the test
\end{enumerate}

\subsection{SWD}
\label{section:SWD}
Single Wire Debug (SWD) is a debug port for pin limited packages, it is a 2+1 pin arrangement vs JTAG 4+1 arrangement. The first pin is used for
clock (\textbf{SWCLK}) with is equivalent to TCK and TMS on JTAG. The other pin (\textbf{SWDIO}) is equivalent to TDI and TDO on JTAG and handles
data transmission. SWD is the ARM standard bi-directional wire protocal and is defined in the ARM Debug interface v5 and can transmit data to and from the
debugger and target board without target-resident code, it is also compatible with all ARM processors. Standard modules provide all normal JTAG debug and test 
functionality as well as power. This system uses synchronous bi-directional transmission, simmilar to $\textrm{I}^2\textrm{C}$.

\subsection{OpenOCD}
\label{section:OpenOCD}
OpenOCD runs a server with is accessible through a terminal on port 4444 and accessible to GDB through port 3333.

\begin{figure}[H]
    \begin{center}
        \begin{tikzpicture}
            % Draw Boxes
            \node(gdb) [boxes] {GDB};
            \node(ide) [boxes, right of=gdb, xshift=3cm] {IDE};
            \node(oocd) [boxes, below of=gdb, fill=red!20, yshift=-1cm] {OpenOCD Server};
            \node(terminal) [boxes, right of=oocd, xshift=3cm] {Terminal\\{\tiny via telnet}};
            \node(jtag) [boxes, below of=oocd, yshift=-1cm] {JTAG or SWD Adapter};
            \node(board) [boxes, below of=jtag, yshift=-1cm] {Board\\{\tiny (may have more than one device)}};

            % Draw Arrows
            \draw[darrow] (gdb) -- (ide) node[midway, above] {MI};
            \draw[darrow] (gdb) -- (oocd) node[midway, right] {:3333};
            \draw[darrow] (oocd) -- (terminal) node[midway, above] {:4444};
            \draw[darrow] (oocd) -- (jtag) node[midway, right] {USB};
            \draw[darrow] (jtag) -- (board) node[midway, right] {JTAG or SWD};
        \end{tikzpicture}
    \end{center}
    \caption{OpenOCD Access structure}
    \label{fiog:openocd}
\end{figure}

\subsection{GDB}
\label{section:GDB}
GDB is a command line debugger in Unix and Linux and is supported under OpenOCD. There is not always a GUI available for the software.\\

The following commands run GDB:
\vspace{0.5cm}

\begin{figure}[H]
\begin{center}
\begin{lstlisting}[language=bash]
gdb $EXE_NAME
gdb -e $EXE_NAME -c $CORE_FILE_NAME
gdb $EXE_NAME --pid=$PROCESS_ID
\end{lstlisting}
\end{center}
\caption{Unix commands to invoke the GDB Debugger}
\label{fig:unix-commands}
\end{figure}
